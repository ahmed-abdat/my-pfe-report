% Annexe A : Code Source du Système

\section{Configuration du Capteur ADXL345}

\begin{lstlisting}[language=C, caption=Initialisation du capteur ADXL345]
#include <Wire.h>
#include "ADXL345.h"

#define ADXL345_ADDRESS 0x53
#define ADXL345_POWER_CTL 0x2D
#define ADXL345_DATA_FORMAT 0x31
#define ADXL345_BW_RATE 0x2C
#define ADXL345_DATAX0 0x32

void setupADXL345() {
    Wire.begin();
    
    // Configuration de la plage de mesure +/-16g
    Wire.beginTransmission(ADXL345_ADDRESS);
    Wire.write(ADXL345_DATA_FORMAT);
    Wire.write(0x0B); // +/-16g, résolution complète
    Wire.endTransmission();
    
    // Configuration du taux d'échantillonnage à 1600Hz
    Wire.beginTransmission(ADXL345_ADDRESS);
    Wire.write(ADXL345_BW_RATE);
    Wire.write(0x0E); // 1600Hz
    Wire.endTransmission();
    
    // Activation de la mesure
    Wire.beginTransmission(ADXL345_ADDRESS);
    Wire.write(ADXL345_POWER_CTL);
    Wire.write(0x08); // Mode mesure
    Wire.endTransmission();
    
    Serial.println("ADXL345 initialisé avec succès");
}

void readAcceleration(int16_t* x, int16_t* y, int16_t* z) {
    uint8_t buffer[6];
    
    Wire.beginTransmission(ADXL345_ADDRESS);
    Wire.write(ADXL345_DATAX0);
    Wire.endTransmission(false);
    
    Wire.requestFrom(ADXL345_ADDRESS, 6, true);
    for(int i = 0; i < 6; i++) {
        buffer[i] = Wire.read();
    }
    
    *x = (buffer[1] << 8) | buffer[0];
    *y = (buffer[3] << 8) | buffer[2];
    *z = (buffer[5] << 8) | buffer[4];
}
\end{lstlisting}

\section{Implémentation de la FFT}

\begin{lstlisting}[language=C, caption=Calcul FFT optimisé pour ESP32]
#include "esp_dsp.h"

#define FFT_SIZE 128
#define SAMPLE_RATE 1600

float wind_hann[FFT_SIZE];
float fft_input[FFT_SIZE * 2];
float fft_output[FFT_SIZE];

void initFFT() {
    // Initialisation de la fenêtre de Hanning
    for (int i = 0; i < FFT_SIZE; i++) {
        wind_hann[i] = 0.5 - 0.5 * cosf(2 * M_PI * i / (FFT_SIZE - 1));
    }
    
    // Initialisation de la bibliothèque DSP
    esp_err_t ret = dsps_fft2r_init_fc32(NULL, CONFIG_DSP_MAX_FFT_SIZE);
    if (ret != ESP_OK) {
        Serial.println("Erreur initialisation FFT");
    }
}

void computeFFT(float* input, float* magnitude) {
    // Application de la fenêtre de Hanning
    for (int i = 0; i < FFT_SIZE; i++) {
        fft_input[i * 2] = input[i] * wind_hann[i];  // Real
        fft_input[i * 2 + 1] = 0;                    // Imaginary
    }
    
    // Calcul de la FFT
    dsps_fft2r_fc32(fft_input, FFT_SIZE);
    
    // Bit-reversal
    dsps_bit_rev_fc32(fft_input, FFT_SIZE);
    
    // Conversion complexe vers réel
    dsps_cplx2reC_fc32(fft_input, FFT_SIZE);
    
    // Calcul de la magnitude
    for (int i = 0; i < FFT_SIZE / 2; i++) {
        float real = fft_input[i * 2];
        float imag = fft_input[i * 2 + 1];
        magnitude[i] = sqrtf(real * real + imag * imag);
    }
}
\end{lstlisting}

\section{Extraction de Caractéristiques}

\begin{lstlisting}[language=C, caption=Extraction des caractéristiques temporelles et fréquentielles]
typedef struct {
    float rms;
    float crest_factor;
    float kurtosis;
    float skewness;
    float peak_frequency;
    float spectral_centroid;
    float energy_bands[8];
} Features;

Features extractFeatures(float* time_data, float* freq_data, int size) {
    Features features = {0};
    
    // RMS (Root Mean Square)
    float sum_squares = 0;
    float mean = 0;
    float max_val = 0;
    
    for (int i = 0; i < size; i++) {
        mean += time_data[i];
        sum_squares += time_data[i] * time_data[i];
        if (fabs(time_data[i]) > max_val) {
            max_val = fabs(time_data[i]);
        }
    }
    
    mean /= size;
    features.rms = sqrtf(sum_squares / size);
    
    // Facteur de crête
    features.crest_factor = max_val / features.rms;
    
    // Kurtosis et Skewness
    float sum_cubed = 0;
    float sum_quad = 0;
    float variance = 0;
    
    for (int i = 0; i < size; i++) {
        float diff = time_data[i] - mean;
        variance += diff * diff;
        sum_cubed += diff * diff * diff;
        sum_quad += diff * diff * diff * diff;
    }
    
    variance /= size;
    float std_dev = sqrtf(variance);
    
    features.skewness = (sum_cubed / size) / powf(std_dev, 3);
    features.kurtosis = (sum_quad / size) / powf(variance, 2) - 3;
    
    // Caractéristiques fréquentielles
    float max_mag = 0;
    int peak_bin = 0;
    float total_energy = 0;
    float weighted_sum = 0;
    
    for (int i = 1; i < FFT_SIZE/2; i++) {
        float mag = freq_data[i];
        total_energy += mag;
        weighted_sum += mag * i;
        
        if (mag > max_mag) {
            max_mag = mag;
            peak_bin = i;
        }
    }
    
    // Fréquence du pic principal
    features.peak_frequency = (peak_bin * SAMPLE_RATE) / FFT_SIZE;
    
    // Centroïde spectral
    features.spectral_centroid = weighted_sum / total_energy;
    
    // Énergie par bandes
    int band_size = (FFT_SIZE/2) / 8;
    for (int b = 0; b < 8; b++) {
        float band_energy = 0;
        for (int i = b * band_size; i < (b+1) * band_size; i++) {
            band_energy += freq_data[i];
        }
        features.energy_bands[b] = band_energy;
    }
    
    return features;
}
\end{lstlisting}

\section{Algorithme K-means pour Détection d'Anomalies}

\begin{lstlisting}[language=C, caption=Implémentation K-means optimisée]
#define NUM_CLUSTERS 5
#define NUM_FEATURES 32
#define ANOMALY_THRESHOLD 2.5

typedef struct {
    float centroid[NUM_FEATURES];
    float radius;
} Cluster;

// Centroïdes et rayons entraînés sur Edge Impulse
Cluster clusters[NUM_CLUSTERS] = {
    // Cluster 0: Fonctionnement normal
    {{0.124f, 0.087f, 0.156f, /* ... */}, 0.125f},
    // Cluster 1: Démarrage
    {{0.156f, 0.102f, 0.189f, /* ... */}, 0.187f},
    // Cluster 2: Basse charge
    {{0.098f, 0.076f, 0.112f, /* ... */}, 0.098f},
    // Cluster 3: Haute charge
    {{0.189f, 0.143f, 0.234f, /* ... */}, 0.156f},
    // Cluster 4: Transitoire
    {{0.145f, 0.091f, 0.167f, /* ... */}, 0.134f}
};

float euclideanDistance(float* a, float* b) {
    float sum = 0;
    for (int i = 0; i < NUM_FEATURES; i++) {
        float diff = a[i] - b[i];
        sum += diff * diff;
    }
    return sqrtf(sum);
}

int detectAnomaly(float* features, float* confidence) {
    float min_distance = FLT_MAX;
    int best_cluster = -1;
    
    // Trouver le cluster le plus proche
    for (int k = 0; k < NUM_CLUSTERS; k++) {
        float dist = euclideanDistance(features, clusters[k].centroid);
        if (dist < min_distance) {
            min_distance = dist;
            best_cluster = k;
        }
    }
    
    // Calculer la confiance
    float threshold = clusters[best_cluster].radius * ANOMALY_THRESHOLD;
    *confidence = min_distance / threshold;
    
    // Détection d'anomalie
    if (min_distance > threshold) {
        return 1; // Anomalie détectée
    }
    
    return 0; // Normal
}
\end{lstlisting}

\section{Communication WiFi et MQTT}

\begin{lstlisting}[language=C, caption=Gestion des alertes via MQTT]
#include <WiFi.h>
#include <PubSubClient.h>
#include <ArduinoJson.h>

const char* ssid = "LAB_FSB_WIFI";
const char* password = "********";
const char* mqtt_server = "192.168.1.100";
const char* mqtt_topic = "maintenance/motor/alerts";

WiFiClient espClient;
PubSubClient mqttClient(espClient);

void setupWiFi() {
    delay(10);
    Serial.print("Connexion à ");
    Serial.println(ssid);
    
    WiFi.begin(ssid, password);
    
    while (WiFi.status() != WL_CONNECTED) {
        delay(500);
        Serial.print(".");
    }
    
    Serial.println("");
    Serial.println("WiFi connecté");
    Serial.println("Adresse IP: ");
    Serial.println(WiFi.localIP());
}

void setupMQTT() {
    mqttClient.setServer(mqtt_server, 1883);
    
    while (!mqttClient.connected()) {
        Serial.print("Connexion MQTT...");
        String clientId = "ESP32-Motor-Monitor";
        
        if (mqttClient.connect(clientId.c_str())) {
            Serial.println("connecté");
        } else {
            Serial.print("échec, rc=");
            Serial.print(mqttClient.state());
            Serial.println(" nouvelle tentative dans 5 secondes");
            delay(5000);
        }
    }
}

void sendAlert(int anomaly_type, float confidence) {
    StaticJsonDocument<256> doc;
    
    doc["timestamp"] = millis();
    doc["device_id"] = "MOTOR_001";
    doc["location"] = "Lab_FSB_BancTest";
    doc["anomaly_detected"] = true;
    doc["anomaly_type"] = anomaly_type;
    doc["confidence"] = confidence;
    doc["severity"] = confidence > 0.8 ? "HIGH" : "MEDIUM";
    
    // Ajout des données de contexte
    JsonObject context = doc.createNestedObject("context");
    context["motor_model"] = "Leroy_Somer_A3S";
    context["rpm"] = 1500;
    context["temperature"] = 45.2;
    
    char buffer[512];
    size_t n = serializeJson(doc, buffer);
    
    if (mqttClient.publish(mqtt_topic, buffer, n)) {
        Serial.println("Alerte envoyée avec succès");
    } else {
        Serial.println("Échec envoi alerte");
    }
}
\end{lstlisting}

\section{Programme Principal}

\begin{lstlisting}[language=C, caption=Boucle principale du système]
#define BUFFER_SIZE 2000
#define ACQUISITION_TIME 1250 // ms

float buffer_x[BUFFER_SIZE];
float buffer_y[BUFFER_SIZE];
float buffer_z[BUFFER_SIZE];
int buffer_index = 0;
unsigned long last_acquisition = 0;

void setup() {
    Serial.begin(115200);
    Serial.println("Système de détection d'anomalies v1.0");
    Serial.println("FSB - Master EEEA 2024-2025");
    
    // Initialisation des composants
    setupADXL345();
    initFFT();
    setupWiFi();
    setupMQTT();
    
    // Configuration des LEDs
    pinMode(LED_GREEN, OUTPUT);
    pinMode(LED_ORANGE, OUTPUT);
    pinMode(LED_RED, OUTPUT);
    
    digitalWrite(LED_GREEN, HIGH);
    Serial.println("Système prêt");
}

void loop() {
    // Acquisition des données
    if (buffer_index < BUFFER_SIZE) {
        if (millis() - last_acquisition >= 1) { // 1ms = 1000Hz
            int16_t x, y, z;
            readAcceleration(&x, &y, &z);
            
            // Conversion en g
            buffer_x[buffer_index] = x * 0.0078f; // ±16g range
            buffer_y[buffer_index] = y * 0.0078f;
            buffer_z[buffer_index] = z * 0.0078f;
            
            buffer_index++;
            last_acquisition = millis();
        }
    } else {
        // Traitement du buffer complet
        unsigned long start_time = micros();
        
        // Calcul FFT sur chaque axe
        float magnitude_x[FFT_SIZE/2];
        float magnitude_y[FFT_SIZE/2];
        float magnitude_z[FFT_SIZE/2];
        
        computeFFT(buffer_x, magnitude_x);
        computeFFT(buffer_y, magnitude_y);
        computeFFT(buffer_z, magnitude_z);
        
        // Extraction des caractéristiques
        Features feat_x = extractFeatures(buffer_x, magnitude_x, BUFFER_SIZE);
        Features feat_y = extractFeatures(buffer_y, magnitude_y, BUFFER_SIZE);
        Features feat_z = extractFeatures(buffer_z, magnitude_z, BUFFER_SIZE);
        
        // Création du vecteur de caractéristiques complet
        float feature_vector[NUM_FEATURES];
        int idx = 0;
        
        // Caractéristiques temporelles
        feature_vector[idx++] = feat_x.rms;
        feature_vector[idx++] = feat_y.rms;
        feature_vector[idx++] = feat_z.rms;
        feature_vector[idx++] = feat_x.crest_factor;
        feature_vector[idx++] = feat_y.crest_factor;
        feature_vector[idx++] = feat_z.crest_factor;
        // ... autres caractéristiques
        
        // Détection d'anomalie
        float confidence;
        int is_anomaly = detectAnomaly(feature_vector, &confidence);
        
        unsigned long processing_time = micros() - start_time;
        
        // Gestion des alertes
        if (is_anomaly) {
            digitalWrite(LED_GREEN, LOW);
            digitalWrite(LED_RED, HIGH);
            
            Serial.printf("ANOMALIE DÉTECTÉE! Confiance: %.2f\n", confidence);
            Serial.printf("Temps de traitement: %lu µs\n", processing_time);
            
            // Envoi alerte MQTT
            sendAlert(1, confidence);
            
            // Buzzer pattern
            tone(BUZZER_PIN, 2000, 500);
        } else {
            digitalWrite(LED_RED, LOW);
            digitalWrite(LED_GREEN, HIGH);
            
            Serial.printf("État normal. Temps: %lu µs\n", processing_time);
        }
        
        // Reset buffer pour nouvelle acquisition
        buffer_index = 0;
    }
    
    // Maintenance connexion MQTT
    if (!mqttClient.connected()) {
        setupMQTT();
    }
    mqttClient.loop();
}
\end{lstlisting}