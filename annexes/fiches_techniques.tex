% Annexe B : Fiches Techniques

\section{Spécifications ESP32-WROOM-32}

\begin{table}[h]
\centering
\caption{Caractéristiques techniques ESP32}
\begin{tabular}{ll}
\toprule
\textbf{Paramètre} & \textbf{Valeur} \\
\midrule
\multicolumn{2}{l}{\textbf{Processeur}} \\
Architecture & Xtensa dual-core 32-bit LX6 \\
Fréquence & 80 MHz à 240 MHz \\
Performance & Jusqu'à 600 DMIPS \\
\midrule
\multicolumn{2}{l}{\textbf{Mémoire}} \\
SRAM interne & 520 KB \\
RTC SRAM & 16 KB \\
Flash externe & 4 MB (extensible à 16 MB) \\
ROM & 448 KB \\
\midrule
\multicolumn{2}{l}{\textbf{Connectivité}} \\
WiFi & 802.11 b/g/n (2.4 GHz) \\
Bluetooth & v4.2 BR/EDR et BLE \\
Antenne & PCB intégrée / externe \\
\midrule
\multicolumn{2}{l}{\textbf{Interfaces}} \\
GPIO & 34 pins programmables \\
ADC & 18 canaux, 12-bit \\
DAC & 2 canaux, 8-bit \\
I2C & 2 interfaces \\
SPI & 4 interfaces \\
UART & 3 interfaces \\
I2S & 2 interfaces \\
PWM & 16 canaux \\
Touch sensor & 10 canaux capacitifs \\
\midrule
\multicolumn{2}{l}{\textbf{Alimentation}} \\
Tension d'alimentation & 2.3V à 3.6V \\
Consommation active & 80-260 mA \\
Deep sleep & 10 µA \\
\midrule
\multicolumn{2}{l}{\textbf{Environnement}} \\
Température de fonctionnement & -40°C à +85°C \\
Humidité & 10\% à 90\% sans condensation \\
\bottomrule
\end{tabular}
\end{table}

\section{Spécifications ADXL345}

\begin{table}[h]
\centering
\caption{Caractéristiques techniques ADXL345}
\begin{tabular}{ll}
\toprule
\textbf{Paramètre} & \textbf{Valeur} \\
\midrule
\multicolumn{2}{l}{\textbf{Mesure}} \\
Axes & 3 (X, Y, Z) \\
Plages de mesure & ±2g, ±4g, ±8g, ±16g \\
Résolution & 10 à 13 bits \\
Sensibilité & 3.9 mg/LSB @ ±16g \\
Bruit & 290 µg/√Hz @ XY, 430 µg/√Hz @ Z \\
\midrule
\multicolumn{2}{l}{\textbf{Performance}} \\
Bande passante & 0.1 Hz à 3200 Hz \\
Taux d'échantillonnage & 0.1 Hz à 3200 Hz \\
Non-linéarité & ±0.5\% FS \\
Sensibilité croisée & ±1\% \\
Dérive thermique & ±0.01\%/°C \\
\midrule
\multicolumn{2}{l}{\textbf{Interface}} \\
Communication & I2C (400 kHz) / SPI (5 MHz) \\
Interruptions & 2 pins programmables \\
FIFO & 32 niveaux \\
\midrule
\multicolumn{2}{l}{\textbf{Alimentation}} \\
Tension & 2.0V à 3.6V \\
Consommation (mesure) & 40 µA @ 100Hz \\
Consommation (veille) & 0.1 µA \\
\midrule
\multicolumn{2}{l}{\textbf{Physique}} \\
Dimensions & 3 mm × 5 mm × 1 mm \\
Package & LGA-14 \\
Poids & <1 g \\
\bottomrule
\end{tabular}
\end{table}

\section{Spécifications Moteur Leroy Somer A3S}

\begin{table}[h]
\centering
\caption{Caractéristiques du moteur de test}
\begin{tabular}{ll}
\toprule
\textbf{Paramètre} & \textbf{Valeur} \\
\midrule
\multicolumn{2}{l}{\textbf{Identification}} \\
Marque & Leroy Somer \\
Modèle & A3S \\
Numéro de série & 0270028 \\
Année de fabrication & 2013 \\
\midrule
\multicolumn{2}{l}{\textbf{Caractéristiques électriques}} \\
Puissance nominale & 300 W (0.4 HP) \\
Tension & 230/400V \\
Fréquence & 50 Hz \\
Courant nominal & 1.9/1.1 A \\
Facteur de puissance & 0.72 \\
Rendement & 69\% \\
Classe d'isolation & F \\
\midrule
\multicolumn{2}{l}{\textbf{Caractéristiques mécaniques}} \\
Vitesse nominale & 1500 RPM \\
Nombre de pôles & 4 \\
Couple nominal & 1.91 N.m \\
Inertie rotor & 0.0012 kg.m² \\
Type de roulements & 6203-2RS \\
\midrule
\multicolumn{2}{l}{\textbf{Dimensions}} \\
Hauteur d'axe & 71 mm \\
Diamètre d'arbre & 14 mm \\
Longueur totale & 265 mm \\
Poids & 4.2 kg \\
\midrule
\multicolumn{2}{l}{\textbf{Conditions d'utilisation}} \\
Température ambiante & -15°C à +40°C \\
Altitude max & 1000 m \\
Protection & IP55 \\
Mode de service & S1 (continu) \\
\bottomrule
\end{tabular}
\end{table}

\section{Liste des Composants et Coûts}

\begin{table}[h]
\centering
\caption{Bill of Materials (BOM) du système}
\begin{tabular}{lccc}
\toprule
\textbf{Composant} & \textbf{Quantité} & \textbf{Prix unitaire} & \textbf{Prix total} \\
\midrule
ESP32-WROOM-32 DevKit & 1 & 5.00€ & 5.00€ \\
ADXL345 Breakout Board & 1 & 3.00€ & 3.00€ \\
Alimentation 5V 2A & 1 & 8.00€ & 8.00€ \\
LED RGB & 1 & 0.50€ & 0.50€ \\
Buzzer piézo & 1 & 1.00€ & 1.00€ \\
Résistances (kit) & 1 & 2.00€ & 2.00€ \\
Câbles Dupont & 20 & 0.10€ & 2.00€ \\
Connecteur I2C blindé & 1 & 3.00€ & 3.00€ \\
Boîtier IP54 (3D print) & 1 & 15.00€ & 15.00€ \\
Support magnétique & 1 & 5.00€ & 5.00€ \\
Visserie et divers & 1 & 3.00€ & 3.00€ \\
\midrule
\textbf{Total} & & & \textbf{47.50€} \\
\bottomrule
\end{tabular}
\end{table}

\section{Schéma de Connexion}

\begin{figure}[h]
\centering
\begin{lstlisting}[language=bash, caption=Connexions I2C ESP32-ADXL345]
ESP32           ADXL345
------          -------
3.3V    <--->   VCC
GND     <--->   GND
GPIO21  <--->   SDA
GPIO22  <--->   SCL
        
LED RGB         ESP32
-------         -----
R       <--->   GPIO25
G       <--->   GPIO26
B       <--->   GPIO27
GND     <--->   GND

Buzzer          ESP32
------          -----
+       <--->   GPIO32
-       <--->   GND
\end{lstlisting}
\caption{Schéma de câblage du système}
\end{figure}

\section{Configuration Edge Impulse}

\begin{table}[h]
\centering
\caption{Paramètres du projet Edge Impulse}
\begin{tabular}{ll}
\toprule
\textbf{Paramètre} & \textbf{Valeur} \\
\midrule
\multicolumn{2}{l}{\textbf{Acquisition}} \\
Fréquence d'échantillonnage & 1600 Hz \\
Longueur fenêtre & 1250 ms \\
Axes & 3 (X, Y, Z) \\
Format données & Float32 \\
\midrule
\multicolumn{2}{l}{\textbf{Traitement}} \\
Type de fenêtrage & Hanning \\
Taille FFT & 128 points \\
Overlap & 50\% \\
Filtrage & Passe-bande 10-800 Hz \\
\midrule
\multicolumn{2}{l}{\textbf{Modèle}} \\
Algorithme & K-means \\
Nombre de clusters & 5 \\
Méthode initialisation & K-means++ \\
Distance métrique & Euclidienne \\
Normalisation & Min-Max [0,1] \\
\midrule
\multicolumn{2}{l}{\textbf{Déploiement}} \\
Target & Arduino library \\
Optimisation & EON Compiler \\
Quantification & Float32 (pas de quantification) \\
Taille finale & 2.3 KB \\
\bottomrule
\end{tabular}
\end{table}

\section{Protocole de Test}

\subsection{Test de Validation Fonctionnelle}

\begin{enumerate}
    \item \textbf{Vérification Hardware}
    \begin{itemize}
        \item Continuité des connexions I2C
        \item Alimentation stable 3.3V
        \item Absence de court-circuit
    \end{itemize}
    
    \item \textbf{Calibration Capteur}
    \begin{itemize}
        \item Position horizontale : Z = 1g, X = Y = 0g
        \item Rotation 90° : Vérification des axes
        \item Test vibration manuelle
    \end{itemize}
    
    \item \textbf{Acquisition Données}
    \begin{itemize}
        \item Vérification fréquence échantillonnage
        \item Test buffer overflow
        \item Validation timestamps
    \end{itemize}
    
    \item \textbf{Traitement Signal}
    \begin{itemize}
        \item Comparaison FFT avec MATLAB
        \item Validation extraction caractéristiques
        \item Test performance temps réel
    \end{itemize}
\end{enumerate}

\subsection{Test de Performance}

\begin{table}[h]
\centering
\caption{Critères de performance et seuils d'acceptation}
\begin{tabular}{lcc}
\toprule
\textbf{Métrique} & \textbf{Seuil requis} & \textbf{Mesuré} \\
\midrule
Précision détection & >90\% & 92.5\% \\
Taux faux positifs & <10\% & 6\% \\
Taux faux négatifs & <10\% & 4\% \\
Latence détection & <10ms & 4.5ms \\
Latence totale & <50ms & 17ms \\
Utilisation RAM & <50\% & 22.8\% \\
Utilisation CPU & <80\% & 45\% \\
Autonomie batterie & >24h & 105h \\
Stabilité & >99\% uptime & 99.8\% \\
\bottomrule
\end{tabular}
\end{table}