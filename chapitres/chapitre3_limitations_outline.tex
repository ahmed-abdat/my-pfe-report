% =========================================================
% CHAPITRE 3 - SECTION LIMITATIONS (OUTLINE)
% =========================================================
% À inclure dans le chapitre 3 (Réalisation et Validation)

\section{Limitations de l'étude}

Cette section présente de manière transparente les limitations de notre validation expérimentale, dans un souci de rigueur scientifique et pour orienter les travaux futurs.

\subsection{Limitations du cadre expérimental}

\begin{itemize}
    \item \textbf{Environnement de laboratoire uniquement} : La validation a été réalisée exclusivement en laboratoire universitaire (FSB), dans des conditions contrôlées qui ne reflètent pas nécessairement la complexité d'un environnement industriel réel (vibrations parasites, variations de température, perturbations électromagnétiques).

    \item \textbf{Moteur unique testé} : L'expérimentation s'est limitée à un seul type de moteur asynchrone (Leroy Somer 0,37\,kW), représentatif des équipements auxiliaires mais ne couvrant pas la diversité des machines industrielles.

    \item \textbf{Gamme de puissance limitée} : Le moteur de 0,37\,kW testé se situe en dessous du seuil de 15\,kW spécifié par la norme ISO 20816-3:2022, limitant la généralisation aux équipements industriels de plus forte puissance.
\end{itemize}

\subsection{Limitations méthodologiques}

\begin{itemize}
    \item \textbf{Défauts simulés uniquement} : Les anomalies ont été introduites artificiellement (masses de déséquilibrage calibrées) plutôt que d'utiliser des défauts naturels évolutifs, ce qui peut affecter la représentativité des signatures vibratoires.

    \item \textbf{Absence de tests longue durée} : L'évaluation s'est concentrée sur des sessions de test ponctuelles sans validation de la stabilité du système sur plusieurs mois d'exploitation continue.

    \item \textbf{Conditions de charge constantes} : Les tests ont été effectués à charge et vitesse constantes, sans explorer les régimes transitoires ou les variations de charge typiques des applications industrielles.
\end{itemize}

\subsection{Limitations techniques}

\begin{itemize}
    \item \textbf{Plage fréquentielle restreinte} : L'ADXL345 limite l'analyse à 1,6\,kHz, insuffisant pour détecter certains défauts haute fréquence (engrenages, roulements à haute vitesse).

    \item \textbf{Axe unique de mesure} : Bien que l'ADXL345 soit tri-axial, l'implémentation actuelle n'exploite qu'un seul axe, limitant la détection de défauts complexes multi-directionnels.

    \item \textbf{Absence de référence industrielle} : Pas de comparaison directe avec des systèmes commerciaux certifiés pour valider la précision absolue des mesures.
\end{itemize}

\subsection{Limitations de déploiement}

\begin{itemize}
    \item \textbf{Validation PME non réalisée} : Aucun test en conditions réelles dans une PME industrielle pour confirmer l'accessibilité et l'utilisabilité revendiquées.

    \item \textbf{Formation utilisateur non évaluée} : L'hypothèse d'une installation en 20 minutes par un technicien non-expert n'a pas été validée empiriquement.

    \item \textbf{Évolutivité non testée} : La scalabilité du système pour surveiller plusieurs machines simultanément n'a pas été explorée.
\end{itemize}

\subsection{Implications pour les travaux futurs}

Ces limitations définissent clairement le périmètre de validité de notre étude : une preuve de concept académique démontrant la faisabilité technique du TinyML pour la détection d'anomalies vibratoires. Les travaux futurs devront adresser ces limitations par :

\begin{enumerate}
    \item Des tests en environnement industriel réel avec défauts naturels
    \item L'extension à une gamme plus large de moteurs (0,37 à 15\,kW)
    \item Une validation longitudinale sur plusieurs mois
    \item Des partenariats avec des PME pour valider l'accessibilité
    \item L'amélioration du système pour l'analyse multi-axiale et haute fréquence
\end{enumerate}