\chapter*{Introduction Générale}
\addcontentsline{toc}{chapter}{Introduction Générale}
\thispagestyle{plain}  % Show page number at bottom center

Dans le contexte de l'Industrie 4.0, la maintenance prédictive s'affirme comme un levier essentiel pour améliorer la disponibilité des équipements et réduire les arrêts non planifiés. Elle prolonge la maintenance conditionnelle en exploitant l'analyse des données de fonctionnement — notamment les vibrations — et des méthodes d'intelligence artificielle pour détecter précocement les comportements anormaux.

Toutefois, le déploiement de solutions complètes reste difficile pour de nombreuses entreprises en raison des coûts, de la complexité d'intégration et de la dépendance à des infrastructures réseau.

Dans ce cadre, le paradigme TinyML — exécuter des modèles d'IA directement sur des microcontrôleurs — offre une alternative intéressante pour analyser au plus près de la machine, avec une faible latence, une meilleure confidentialité et des coûts contenus.

Ce mémoire présente une étude sur le développement d'un système de détection d'anomalies pour moteurs industriels basé sur TinyML, explorant le potentiel de cette technologie pour rendre la maintenance prédictive plus accessible aux PME industrielles. Notre étude explore l'utilisation d'algorithmes d'apprentissage non-supervisé sur microcontrôleur embarqué. La validation expérimentale, réalisée en laboratoire universitaire, permet d'évaluer la faisabilité technique de cette approche.

\textbf{Note méthodologique :} Ce travail se concentre sur la \textit{faisabilité technique} d'une implémentation TinyML pour la détection d'anomalies vibratoires. Les estimations économiques citées visent à contextualiser l'enjeu stratégique, mais ne constituent pas l'objet principal de validation de ce PFE. Une évaluation économique complète (calcul de ROI, analyse coût-bénéfice sur équipements réels) nécessiterait un déploiement industriel à long terme dépassant le cadre d'un projet de fin d'études.

Le présent document est structuré comme suit :

Le \textbf{Chapitre 1} présente l'état de l'art, établissant les bases théoriques nécessaires : l'évolution de la maintenance industrielle, les concepts fondamentaux de l'analyse vibratoire et du TinyML, ainsi qu'une revue critique de la littérature récente pour identifier les lacunes et positionner notre contribution.

Les chapitres suivants, en cours de rédaction, détailleront la conception du système, sa réalisation pratique et les résultats expérimentaux obtenus.